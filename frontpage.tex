%%%%%%%%%%%%%%%%%%%%%%%%%%%%%%%%%%%%%%%%%%%%%%%%%%%%%%%%%%%%%%%%%%
% page de garde : Université de Nantes
% EDSTIM
\begin{titlepage}
\thispagestyle{empty}

\begin{center}
\strong{%
\Large{%
Université de Nantes\\
École doctorale\\
« Sciences et technologies de l'information et des mathématiques »}}

\vspace{.5cm}

\large{Année 2008}

\vspace{.5cm}

\strong{\Large{Thèse de doctorat de l'université de Nantes}}

Spécialité : automatique et informatique appliquée

\vspace{.5cm}

\emph{présentée et soutenue publiquement par}

\vspace{.5cm}

\strong{\Large{Stéphane Péchard}}

\vspace{.5cm}

\emph{le 2 octobre 2008}

\emph{à l'École polytechnique de l'université de Nantes}

\vspace{1cm}

\huge{Qualité d'usage en télévision haute définition : \\évaluations subjectives et métriques objectives}

\vspace{1cm}

\strong{\large Jury}
\end{center}

\begin{tabular}{lll} % TODO à améliorer : pourquoi pas les 2 ? en tout cas si une seule : les etablissements
Rapporteurs
	& M. François-Xavier Coudoux 			& \emph{Professeur, université de Valenciennes}\\
	& M. Michel Jourlin						& \emph{Professeur, université de Saint-Étienne}\\
Examinateurs
	& M. Joseph Ronsin						& \emph{Professeur, INSA Rennes}\\
	& Mme	Sheila Hemami					& \emph{Professor, Cornell University, États-Unis}\\
	& Mme Patricia Ladret					& \emph{Maître de conférences, Polytech'Grenoble}\\
	& M. Dominique Barba 					& \emph{Professeur émérite, Polytech'Nantes}\\
	& M. Patrick Le Callet 					& \emph{Professeur, Polytech'Nantes}\\
Membre invité
	& M. Ricardo Pastrana-Vidal 			& \emph{Ingénieur, France Télécom R\&D, Rennes}\\
\end{tabular}

\vspace{.5cm}

\hrule

\vspace{.2cm}

\noindent Directeur de thèse : Dominique Barba\\
Co-encadrant : Patrick Le Callet\\
Institut de recherche en communications et cybernétique de Nantes \hspace{\stretch{1}} N\degre ED 503--004
\end{titlepage}
