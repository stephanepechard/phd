\chapter*{Bilan des tests subjectifs réalisés} \addcontentsline{toc}{chapter}{Bilan des tests subjectifs réalisés} % TODO à vérifier
Cette première partie a abordé la question de la qualité visuelle en télévision haute définition d'un point de vue subjectif. Nous l'avons vu, cet aspect nécessite de lourdes expérimentations. À titre informatif, voici quelques chiffres caractérisant, chacun à leur façon, l'ensemble des tests que nous avons réalisés. Nous avons utilisé :
\begin{itemize}
\item une salle de test dédiée et normalisée selon les recommandations internationales~\cite{itu-bt500-11,itu-bt710-4} ;
\item une chaine complète de lecture de séquences de télévision haute définition non compressées comprenant deux écrans haute définition, un lecteur temps-réel et un serveur de stockage ;
\item près de deux téraoctets de séquences vidéos ;
\item plus de 750 jours cumulés de codage H.264 avec le codeur de référence~\cite{h264-jm} ;
\item une interface de notation pour chacune des trois méthodologies utilisées ;
\item 20 sessions SAMVIQ, 4 sessions ACR et 2 sessions de préférence ;
\item 600 séances de test pour environ 300 heures d'évaluation subjective de la qualité ;
\item une base de plus de 200 observateurs uniques, dont l'acuité est vérifiée par le test de Monoyer~\cite{monoyer-plates} et l'absence de daltonisme par les tests d'Ishihara~\cite{ishihara-plates}.
\end{itemize}

Les deux écrans de résolution 1920\texttimes1080 utilisés étaient un CRT JVC DT-V 1910CG et un LCD Philips T370HW01. Le premier, de type CRT, a une hauteur d'image de 20,5 cm, ce qui donne une distance d'observation de 61,5 cm en se plaçant à trois fois la hauteur de l'écran. Le second écran est un LCD de 46 cm de hauteur, correspondant à une distance d'observation normalisée de 138 cm. Le lecteur temps-réel de contenu haute définition 1080i non compressé était un système Dorémi V1-UHD.


% photo ?
