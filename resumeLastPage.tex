\paragraph*{Résumé}
La télévision haute définition (TVHD) est le nouveau système de diffusion sensé apporter une plus grande immersion de l'observateur dans l'action et une qualité d'usage supérieure à celle de la télévision standard (TVSD). Afin de remplir ces exigences, les diffuseurs ont besoin de techniques de mesure de qualité adaptées. Le premier objectif de cette thèse est de nous éclairer sur l'impact de la transition de la TVSD vers la TVHD sur la qualité visuelle. Le second est de proposer des métriques objectives de qualité adaptées à la TVHD. Nous avons tout d'abord étudié plusieurs aspects de la qualité visuelle dans ce contexte. Cela nous a notamment permis de sélectionner une méthodologie d'évaluation subjective de la qualité pour constituer une base de séquences dégradées par un codage H.264. Puis, nous avons proposé une méthodologie d'évaluation de l'impact d'un système dégradant sur la qualité visuelle. Elle est utilisée dans les trois critères objectifs de qualité présentés par la suite. Elle fournit au premier une analyse du contenu permettant de modéliser la qualité visuelle d'une séquence. Le second critère cumule des fonctions de gêne locales en une note de qualité globale grâce à des modèles validés par la méthodologie. Le dernier exploite la notion de tubes spatio-temporels introduits par la méthodologie. Des caractéristiques y sont mesurées pour les séquences de référence et dégradée. Le cumul des différences entre ces caractéristiques fournit une note de qualité finale. Alors que le second critère n'offre pas de résultats convaincants, les deux autres permettent une prédiction des notes subjectives plus performante que celle de deux critères connus de la littérature.

\paragraph*{Mot-clés :} télévision haute définition, qualité subjective de la vidéo, codage H.264.

\pagestyle{empty}

\begin{otherlanguage}{english}
\paragraph*{Abstract}
High Definition Television (HDTV) is the new broadcasting system expected to offer a deeper immersion in action to the observer and a higher quality of experience than standard television (SDTV). In order to meet these requirements, broadcasters need better adapted quality measurement techniques. The first objective of this thesis is to clarify the impact on visual quality of the transition from SDTV to HDTV. The second is to propose objective quality metrics adapted to HDTV. We first studied some aspects of visual quality in this context. This allowed us to select a subjective assessment methodology to build a H.264-distorted sequence base. Then, we proposed a methodology to assess the impact of a distorting system on visual quality, to be used in the three objective metrics presented. Applied to the first one, it provides a content analysis used to model the visual quality of a sequence. The second one pools local distortion functions in a global quality score thanks to validated models. The last one uses the notion of spatio-temporal tubes introduced by the methodology. Features are extracted in these tubes for reference and distorted sequences. The pooling of features differences provides a final quality score. While the second metric obtains unconvincing results, the two others allow a more effective subjective score prediction than two other known metrics from the literature.

\paragraph*{Keywords:} High Definition Television, video subjective quality, H.264 coding.

\end{otherlanguage}

% \ornementChapitre
