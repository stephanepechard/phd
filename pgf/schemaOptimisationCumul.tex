% schéma de l'optimisation du cumul

\begin{tikzpicture}[node distance = 3cm, auto]
    % Place nodes
	\node (MdID) {$\Md(l,T)$};
	\node [right of=MdID, node distance = 1.25cm] (MdVide) {};
	\node [below of=MdVide, node distance = 1.25cm] (belowMdVide) {};
	\node [action, right of=MdID, text width=5em, node distance = 2.5cm] (cumulID) {cumul inter-distorsions};
	\node [action, right of=cumulID, text width=5em, node distance = 3.5cm] (cumulTemp) {cumul temporel};
	\node [action, right of=cumulTemp, text width=5em] (fAjust) {fonction d'ajustement};
	\node [right of=fAjust] (rmse) {$\RMSE(\MOS, \MOSp)$};
	\node [right of=fAjust, node distance = 1.5cm] (rmseVide) {};
	\node [below of=rmseVide, node distance = 1.25cm] (belowrmseVide) {};

    % Draw edges
	\path [fleche] (MdID) -- (cumulID);
    \path [fleche] (cumulID) -- node {$\Md(T)$} (cumulTemp);
	\path [fleche] (cumulTemp) -- node {$\Md$} (fAjust);
	\path [fleche] (fAjust) -- (rmse);
	\path [fleche] (10.2,0) -- (10.2, -1) -- (1,-1) -- (1,0); % TODO à revoir
\end{tikzpicture}
