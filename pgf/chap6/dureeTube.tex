% schéma du tube dans une image

\draw[help lines, dashed] (1.1,-3) -- (21.8,-3);
\draw[help lines, dashed] (2.1,-2.5) -- (22.9,-2.5);
\draw[help lines, dashed] (2.1,-1) -- (22.9,-1);

\filldraw[help lines, fill=blue!20] (1.1,-3) -- (2.1,-2.5) --  (2.1,-1) -- (1.1,-1.5) -- cycle;
\filldraw[help lines, fill=blue!20] (6.3,-3) -- (7.3,-2.5) --  (7.3,-1) -- (6.3,-1.5) -- cycle;
\fill[fill=blue!20] (11.5,-3) -- (12.5,-2.5) --  (12.5,-1) -- (11.5,-1.5) -- cycle;
\filldraw[help lines, fill=blue!20] (16.7,-3) -- (17.7,-2.5) --  (17.7,-1) -- (16.7,-1.5) -- cycle;
\filldraw[help lines, fill=blue!20] (21.9,-3) -- (22.9,-2.5) --  (22.9,-1) -- (21.9,-1.5) -- cycle;

\draw[help lines, dashed] (1.1,-1.5) -- (21.9,-1.5);


% \begin{tikzpicture}[scale=0.5]
\foreach \i in {0.25,0.75,1.25,1.75} % grille sur l'image centrale
{
	\draw (10+2*\i,\i) -- (10+2*\i,-6+\i);
	\draw (10,-3*\i) -- (14,2-3*\i);

	\foreach \j in {0,1,3,4} % grille sur les autres images
	{
		\draw (5*\j + 2*\i, \i) -- (5*\j + 2*\i, -6 + \i);
		\draw (5*\j,-3*\i) -- (5*\j + 4,2-3*\i);
	}
}

\draw (2,-6.5) node{$i-2$};
\draw (7,-6.5) node{$i-1$};
\draw (12,-6.5) node{image $i$};
\draw (17,-6.5) node{$i+1$};
\draw (22,-6.5) node{$i+2$};

\draw[<->] (12,2) -- (22,2) node[above,pos=0.5] {80 ms};
\draw[<->] (2,3) -- (22,3) node[above,pos=0.5] {tube};

% \end{tikzpicture}
